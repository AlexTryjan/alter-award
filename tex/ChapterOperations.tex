\chapter{Chapter Operations}

  The Beta Nu Chapter has an extremely well organized and documented program for chapter operations. All of our executive officers, elected appointed positions, and committee description resources are posted on our shared Google Drive for public chapter use.
  
  \section*{Elections and Appointments}
    Beta Nu's elections take place over the course of three weeks. During the first week, nominations are accepted for all elected positions. During the second week, Executive Council elections are held, while non-executive positions are elected in the third week. Before every position is elected, the floor is opened for nominations one final time. During the election, every candidate has a set speaking and question-answer time. Our bylaws stipulate a minimum time, and by custom the candidates are limited to:
    
    \begin{itemize}
    	\item President 
	  7 minutes to speak, 5 minutes to answer questions. All unused speech time
	  can be used for questions.
	  
	\item Other EC Positions
	  5 minutes to speak, 4 minutes to answer questions. All unused speech time can
	  be used for questions.
	  
	\item Non EC Positions
	  3 minutes to speak, 2 minutes to answer questions. All unused speech time can
	  be used for questions.
    \end{itemize}
    
    After each candidate speaks, the candidates withdraw while the chapter has discussion. Once again, the bylaws give minimum times for discussion, and by custom discussions are limited to 15 minutes for President, 10 minutes for other EC positions, and 5 minutes for non-EC positions. Votes are done via a secret ballot collected by the Second Guard and the Chaplain. Any brother interested in holding an appointed position is asked to submit a letter of intent to the President detailing his plans for the position and any qualifications. The President then selects who he wants to fill the position - in conjunction with the Executive Council - and the chapter approves it during the final chapter of the semester. Installation of officers is held during the final chapter of the semester. 
    
    \section*{Executive Council}
      The Executive Council is composed of the President, Vice-President, Vice President of Health and Safety, Secretary, Treasurer, Marshal, Recruitment Chair, Scholarship Chair, and EC Member at Large. The Member at Large is a non-voting member of EC who has previously held an EC position; he is the only appointed member of the EC. The Council meets once a week to discuss matters of chapter operations including scholarship, finance, ritual activities, goals, and standards. EC has broad powers to take matters into their own hands, but everything must be approved by the chapter at the next regularly scheduled chapter meeting.
      
      \subsection*{President}
	Beta Nu's Fall 2014 President, Daniel Snow, was the first President in recent chapter history to operate on a calendar year basis rather than an academic one. As such, he was able to follow through with some of his goals through from the previous semester into the new academic year. These included such internal chapter goals as increasing accountability through increased attention of the executive council to each other's duties and actions, as well as improving his direction of the duties of general membership within the chapter as a whole. He also worked to improve the chapter culture surrounding leadership by encouraging people with concerns or ideas to talk to the officer with whom they may pertain, rather than solely approaching the President or Vice President. In addition, he continued to work closely with the Vice President of Health and Safety to keep a close eye on the mental health of the chapter. Externally, his goals included increasing campus involvement within and outside of Greek Life, as well as helping the Vice President of Health and Safety develop Beta Nu's first annual Mental Health Awareness Week. Moving forward, the newly elected President, Aditya Rengaswamy, is working to improve alumni engagement and develop a scholarship fund for brothers in need; he also desires to improve membership development, scholarship, goal setting, and accountability.
	
      \subsection*{Vice President}
	The Vice President's main roles are internal. He heads both our Executive Council and Committee Chair Committee meetings every week. Our first full-year Vice President, Austin Hacker, has been working to improve the position on a variety of fronts. He has increased transparency of the executive council as well as attendance of non-EC members in an effort to make sure the broader chapter has the ability to weigh in on more decisions. He has increased officer accountability by meeting with each officer on a weekly basis. Austin has held a bylaw revision committee, personally going through our local bylaws and updating them to make them more relevant to the current state of the chapter.
	
      \subsection*{Vice President of Health and Safety}
	Our Vice-President of Health and Safety has accomplished remarkable feats during his first year of the position. James Fitzpatrick, our VPHS in the fall, ran the first annual Mental Health Awareness Week (MHA Week). After examining problems we felt the campus and the chapter faced, we decided that advocating for mental health was the best direction. The events were a big success, and Mental Health Awareness will continue to be our main mode of advocacy and philanthropy. After MHA Week, we inspired a number of student groups to form, including Active Minds and an on-campus affiliate of NAMI. Furthermore, other student leaders have now begun advocating for mental health, and our VPHS now regularly meets with officers of other Greek Chapters to help them make similar initiatives. \\

	Our VPHS plays many other roles as well. He has taken over some of the responsibilities our Chaplain used to have, including but not limited to counseling brothers, acting as a mediator during conflicts, and being a person to simply speak with. This semester, the new VPHS - Karthik Mohanarangan - has updated the first aid kit, added commonly needed medication to house stores, and posted crisis management steps throughout the house. \\

	Brother Mohanarangan has also spent a large portion of his time networking with campus officials and building relationships. Our current VPHS knows individuals at University Counseling Services, the Greek Life Office, and participates in Greek Life's Culture of Care Initiative.
	
      \subsection*{Secretary}
	The Secretary's primary duties are to record the actions of the chapter, facilitate communication to the brothers, and handle all communication to the Grand Chapter and CWRU's Greek Life Office. During chapter, he records the minutes and attendance. Additionally, he receives and organizes officer reports, excuses, and other information/communication requests. He performs a broad array of organizational maintenance, including the mail closet, chapter website, FTP server, Google Drive, and the ``God Calendar''- so named because if an event isn't on the God Calendar, it doesn't exist. \\

	One of the programs our Secretary has brought back is TWIOX- ``This Week in Theta Chi''. This is a newsletter that is sent out to the email list that collects and details different chapter happenings and brotherhood events during the next week. The newsletter often contains humorous additives. \\

	Lastly, the Secretary handles awards, such as Alter Award and the analogous campus Greek Life Pytte Cup - each of which are major endeavors.
	
      \subsection*{Treasurer}
	The role of our Treasurer is very traditional. He manages the budget, communicates with the chapter Housing Corp., and any entity the chapter deals with on a financial basis. He, in tandem with the chapter, sets the budget, along with prices for dues and rent. He oversees payment from each brother, utilizes  OmegaFi, and he handles any payment deferments that brothers request. He ensures our taxes are paid accurately and in a timely manner, and he keeps an emergency fund in case unforeseen events unfold. 
      
      \subsection*{Marshal}
	The Marshal's job is to introduce each new candidate class to the fraternity, teach them our ways and history, and make them into men who are capable brothers and leaders. Each Marshal has substantial preference in how he chooses to run the position. Every semester, the Marshal writes up a plan which the chapter then reviews and votes on. This past fall, Brother Rengaswamy was Marshal and made it his goal to first: make the candidates feel at home in joining the brotherhood, and second: ensure the candidates got each of their ``P's`` completed by Initiation: the Philanthropy event, social Party, and Pledge Plaque. This semester, Peyton Turner has taken in the candidates and set out a number of goals that we hope will have a lasting effect beyond this semester. In addition to running Initiation, he has replaced aging or missing materials that help set the mood of the ceremony; he is also working on replacing robes, in order to make Initiation as profound as possible. 
	
      \subsection*{Recruitment Chair}
	The Recruitment Chairman's main role in the chapter is to plan and execute Rush as well as organize 365 recruitment events. During fall rush, our Recruitment Chair was away due to work. Due to successful long-term planning as well as a competent Rush Chair who took over, we initiated seven men, a typical amount for us at Case Western. Michael Bending led the spring rush as the next rush chairman. He worked in particular to keep constant contact with rushees and made sure brothers were inviting them to future rush events. Consequently, we took in another seven men, which is above our average for a spring rush. Now, the Recruitment Chair is working on additional 365 events and has finished planning out the next rush. We look forward to seeing the results of this effort for next fall, and expect a pledge class of 10.
	
      \subsection*{Scholarship Chair}
	The Scholarship Chair works to aid the academic success of brothers - particularly those who are struggling - as well as celebrate those who are doing well. Last semester, the Scholarship Chair, Kris Sabatini, took great strides in changing the environment of the house as a study environment. \\

	He enforced the study room as a dedicated academic space rather than simply another room for people to socialize. In addition, he started utilizing the Scholarship Committee in order to run study tables. Every school weekday this year, for two hours in the evening, the first floor living room has become a moderated study room, providing brothers an opportunity to get their work done in a large quiet space. \\

	Each semester, the Scholarship Chair holds a scholarship dinner. During the fall, the dinner is co-hosted with a sorority; we invited professors, hosted speakers, and celebrated the brothers who have performed exceptionally. In the spring, the dinner is traditionally held at the house. \\

	The Scholarship Chair puts brothers below the 3.0 standard on scholarship contracts, giving them individualized requirements. These requirements can include mandatory study hours and study table attendance, weekly reports on grades/progress, and a maintained calendar of assignments. Through the work of the past Scholarship Chair, as well as the current Scholarship Chair, Hunter Yevincy, we have raised our chapter GPA from last semester as well as the year before. As of now, we only have 2 brothers on academic deficiency. 
	
    \section*{Other Officers}
	The other elected officers are the Historian, Chaplain, First Guard, Second Guard, Assistant Treasurer, and the Standards Board Justices. Our appointed positions include the Social Chair, Alumni Relations Chair, Member Development Chair, House Manager, Assistant House Manager, Detail Manager, Philanthropy and Service Chairman, Public Relations Chairman, Risk Manager, IFC Representative, Librarian, OX Roast Chair, Greek Week Chair, Food Steward, and the Athletics Chair. Appointed positions may be created at will by the President to accomplish certain tasks.
	
    \section*{Local Bylaws}
	Beta Nu's local by-laws are stored on the publicly shared Google Drive our chapter uses for data management. We have held several by-law revision roundtables to facilitate the relevancy of these documents; most recently, we plan to update the scholarship by-laws as per the revisions by current Scholarship Chair, Hunter Yevincy. 
	
    \section*{Goals and Retreats}
	Chapter goals are set by the chapter during the semesterly retreat. During the Retreat, topics ranging from recruitment advice to chapter finance management are discussed. Most recently, we created a new survey, the \textit{Your Time at Theta Chi} survey, to gather comprehensive information about brother interests, challenges, and goals for the chapter.
	
    \section*{Committee}
	Beta Nu has three different committee types: Operational, Standing, and Ad-Hoc committees. \\
	
	Standing committees meet every other week and consist of the Recruitment Committee, PR Committee, Service Committee, and the Social Committee. Operational committees meet monthly and consist of the Alumni Relations Committee, Member Development Committee, and Scholarship Committee. Ad-hoc committees meet as necessary, and include such things as the VPHS Committee, the budget setting committee, and bylaws revision committee. All committee chairs meet weekly for the Committee Chair Committee, which is headed by the Vice-President, in order to discuss the week’s progress and to collaborate with each other.
	
	\begin{enumerate}
	  \item The Recruitment Committee is chaired by the Recruitment Chair, and it assists him in planning recruitment for the next semester. This includes 365 recruitment events, the recruitment calendar, PR plans, and planning recruitment workshops.
		
	  \item The Public Relations (PR) Committee is chaired by the PR Chair and assists in planning PR campaigns for our events. They are also in charge of managing our brand on campus. 
		
	  \item The Philanthropy and Service Committee helps the chair organize and plan service events for the semester.
		
	  \item The Social Committee helps the chair organize social events. They work closely with the PR committee to advertise events, the Recruitment Chair to plan 365 events, and the Service Committee to plan service mixers. 
		
	  \item The Alumni Relations Committee is headed by the Alumni Relations Chair and is responsible for alumni outreach. This includes organizing the annual Christmas party, Founder’s Day celebrations, and the newsletter. It is largely through their efforts that the chapter has such a strong showing in the Roster Book Rally.
		
	  \item The Member Development Committee, headed by the Membership Development Chair, is in charge of planning membership development mini-sessions in chapter and hosting a variety of programs to help brothers become better men.
		
	  \item The Scholarship Committee is focused on chapter grades. They assist the Scholarship Chair in providing help for struggling members. They also help plan the scholarship recognition dinner, and invite campus resources to speak with the chapter.
	\end{enumerate}
    
    \section*{Standards Board}
      The chapter maintains a fully functional Standards Board that consists of an Arbiter, Scribe, Parliamentarian, and seven Justices. Justices cannot be members of the Executive Council. \\

      The role of Standards Board is to ensure compliance with the International Bylaws, Local Bylaws, Ritual, and campus rules. Any brother may bring any other brother to Standards, at which point the Board mediates the dispute and issues sanctions with an eye toward helping to resolve strife rather than perpetuating it. \\

      Far more commonly, the Standards Board is used to recognize brothers. They meet at least monthly to issue awards such as Brother of the Month, Officer of the Month, and Alumnus of the Month. These awards recognize the brothers who have made outstanding contributions to the chapter and exemplify the ideals of Theta Chi. Others awards may be given at the discretion of the board. A copy of our local bylaws is included within the appendices. \\
      
      One unique endeavor that we did last semester to improve our standards board process was to invite the Associate Director of Greek Life, Amie Jackson, to discuss how we can improve our standards board. She provided valuable advice. This included tactics on when to consider lowering the frequency of meetings, what kinds of recognition to include for brothers as facilitated by the board, and the need to create more physical awards when possible. She also urged us to continue to review our local bylaws periodically to ensure they match the current opinions of the chapter. Our most recent bylaw revision and approval was in 2013, and we plan on reviewing them again this year. We typically review bylaws every 2 years in the Fall semester.
      
    \section*{Documentation and Transitions}
      When transitioning, every incoming officer is required to meet with their predecessor to discuss the specific details of the position, set goals, and get any tips. Both officers then meet with the Vice-President where they discuss their transitions. Every officer also has a notebook that is passed down that details the day to day minutia of the officers. Officers also keep any documents relevant to their position the chapter server (similar to Dropbox). Every relevant document going back as far as 2001 is kept on this server for future usage.
      
    \section*{Internal Communications}
      The chapter maintains an email list that all brothers are a part of. Reminders and announcements are sent to this list, and it is integral to our chapter operations. We also keep an online calendar lovingly called ''The God Calendar.`` All brothers can create or edit events. Common events include chapter meetings, campus events, and so on. Additionally, the Secretary has brought back TWIOX- \textit{This Week in Theta Chi}. We believe a major reason for the improvement in attendance of a variety of events is due to the revival of TWIOX.